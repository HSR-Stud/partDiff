\subsection{Elliptische PDGL}
$\Delta u=f\qquad \omega=\{(x,y)|y\geq 0\},\quad u(x,y)=ay$

\textbf{Satz:} Wenn $\Omega$ beschränkt und zusammenhängend, dann ist die Lösung u immer eindeutig.\\

\textbf{Beweis:} Annahme: $u=u_1-u_2$\\
Einsetzen: $\Delta u_1 - \Delta u_2=f-f=0$\\
$\left.(u_1-u_2)\right|_{\partial \omega}=g-g=0$\\
$\Delta u=0 \qquad \left.u\right|_{\partial\Omega}=0$\\
Falls $u=0$ eine Lösung, dann gibt es nur eine Lösung.

\subsubsection{Maximumprinzip} 

Wenn gilt $\Delta u=0$, dann befinden sich die Extrema (Maxima und Minima der Funktion) auf dem Rand $\partial\Omega$.

\subsubsection{Beispiel (Übungslösungen)}
Eine elliptische PDGL wie $\Delta u = c$ hat mit der vorgegebenen
Dirichlet-Randwerten nur eine Lösung. Zur Erinnerung: Der Grund war das
Maximum-Prinzip. Gäbe es nämlich eine zweite Lösung $\bar v(r,\phi)$ mit
gleichen Randwerten, wäre $v - \bar v$ eine Lösung der Gleichung $\Delta (v -
\bar v) = 0$ also harmonische Funktion. Die Randwerte von $v - \bar v$ sind 0. Da eine
harmonische Funktion das Maximum auf dem Rand annimmt ist $v - \bar v = 0$ die
Lösung ist also eindeutig.

\subsubsection{Greensche Funktion} 

Eine elliptische PDGL wird mittels Inversion von $\Delta$ gelöst. Dieser Umkehr geschieht mittels Greenscher Funktion, welche die Umkehrfunktion $\Delta$ ist\qquad $\Delta$: Laplace-Operator.

$u(x)=\int\limits_\Omega{\sigma(x,\xi)f(\xi)d\xi}+\int\limits_\Omega{h(x,\xi)f(\xi)d\xi}\qquad \sigma(x,\xi)=
\begin{cases}
	\frac 12|x-\xi| & n=1\\ 
	\frac 1{2\pi}\log|x-\xi| & n=2\\
	-\frac 1{4\pi}\frac{1}{|x-\xi|} & n=3\\
	\frac {1}{(2-n)\mu(S^{n-1})}|x-\xi|^{2-n} & n\geq 3\\
\end{cases}$\\


Greensche Funktion: $G(x,\xi)=\sigma(x,\xi)+h(x,\xi)$

Satz: Ist $\Omega$ ein Gebiet, auf dem das Dirichlet Problem eindeutig lösbar ist, dann gibt es eine Funktion $G(x,\xi)$, welche als Funktion von x die Gleichung

$\Delta G(x,\xi)=\delta(x-\xi)$

löst mit homogenen Randbedingungen.
Lösung: $u(x)=\int\limits_{\Omega}^{}{G(x,\xi)f(\xi)d\xi}+\int\limits_{\partial\Omega}g(\xi)\cdot\grad{\xi}G(x,\xi)d\eta\qquad \eta:\text{ Normale von }\partial G$

\subsubsection{Mittelwerteigenschaft harmonischer Funktionen}

$\Delta h=0$\qquad Mittelwerteigenschaft:\qquad $h(x)=
\begin{cases}
	\frac{h(x+\delta)+h(x-\delta)}{2}& n=1 \\ 
	\frac 1{2\pi r} \int\limits_{S_r^1}{h(x+\xi)d\xi} & n=2\\
	\frac 1{4\pi r^2} \int\limits_{S_r^2}{h(x+\xi)d\xi} & n=3\\
\end{cases}$