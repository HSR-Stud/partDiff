\section{FEM}

Der Vektorraum $\mathbb{V}$ hat undendlich viele Dimensionen. Falls wir n unabh�ngige Funktionen $v_1,\ldots,v_n$ w�hlen, dann spannen die Funktionen $a_1\cdot v_1(x)+\ldots+a_n\cdot v_n(x)$ einen n dimensionalen Teilraum $\mathbb{V}^{(n)}$  von $\mathbb{V}$ auf. Dabei gilt:\\

$\boxed{\tilde{u}^{(n)}=a_1\cdot v_1(x)+\ldots+a_n\cdot v_n(x)}$
\subsection{Das Verfahren von Ritz}
\textbf{Ritzsche Matrize: }
$R^{(n)}=\begin{bmatrix}
	R_{1,1}& R_{1,2}&\cdots\\
	R_{2,1}& R_{2,2}&\cdots\\
	\vdots & \vdots &\ddots\\
\end{bmatrix}$ \qquad mit \qquad $R_{j,k}^{(n)}=\int\limits_{0}^{1}{v_j'(x)\cdot v_k'(x) dx}$\\
\textbf{Ritzscher Vektor: } 
$r^{(n)}=\begin{bmatrix}
	r_1\\
	r_2\\
	\vdots\\
\end{bmatrix}$ \qquad mit \qquad $r_{k}^{(n)}=\int\limits_{0}^{1}{f(x)\cdot v_k(x) dx}$\\

\textbf{L�sung nach Ritz:} $R^{(n)}\cdot a=r^{(n)}\qquad \Rightarrow \qquad a=\left\{R^{(n)}\right\}^{-1}\cdot r^{(n)}$
\subsection{Das Verfahren von Galerkin}
\textbf{Galerksche Matrize: }
$G^{(n)}=\begin{bmatrix}
	G_{1,1}& G_{1,2}&\cdots\\
	G_{2,1}& G_{2,2}&\cdots\\
	\vdots & \vdots &\ddots\\
\end{bmatrix}$ \qquad mit \qquad $G_{j,k}^{(n)}=\int\limits_{0}^{1}{v_j''(x)\cdot v_k(x) dx}$\\
\textbf{Galerkscher Vektor: } 
$g^{(n)}=\begin{bmatrix}
	g_1\\
	g_2\\
	\vdots\\
\end{bmatrix}$ \qquad mit \qquad $g_{k}^{(n)}=\int\limits_{0}^{1}{f(x)\cdot v_k(x) dx}$\\

\textbf{L�sung nach Galerkin:} $G^{(n)}\cdot a+g^{(n)}=0\qquad \Rightarrow \qquad a=\textcolor{red}{\mathbf{-}}\left\{G^{(n)}\right\}^{-1}\cdot g^{(n)}$


\subsection{Gewichtete Residuen (Bereichskollokation)}

Gewichtungsfunktionen: $\{w_1(x),\ldots,w_n(x)\}$

\textbf{Matrize (gewichtete Residuen): }
$M^{(n)}=\begin{bmatrix}
	M_{1,1}& M_{1,2}&\cdots\\
	M_{2,1}& M_{2,2}&\cdots\\
	\vdots & \vdots &\ddots\\
\end{bmatrix}$ \qquad mit \qquad $M_{j,k}^{(n)}=\int\limits_{0}^{1}{v_j''(x)\cdot w_k(x) dx}$\\
\textbf{Vektor (gewichtete Residuen): } 
$m^{(n)}=\begin{bmatrix}
	m_1\\
	m_2\\
	\vdots\\
\end{bmatrix}$ \qquad mit \qquad $m_{k}^{(n)}=\int\limits_{0}^{1}{f(x)\cdot w_k(x) dx}$\\

\textbf{L�sung der gewichteten Residuen:} $M^{(n)}\cdot a+m^{(n)}=0\qquad \Rightarrow \qquad a=\textcolor{red}{\mathbf{-}}\left\{M^{(n)}\right\}^{-1}\cdot m^{(n)}$

\subsection{Punktkollokation}
Im Sinne einer Punktkollokation werden n St�tzstellen im Intervall von [0,1] gew�hlt.\\

$\begin{bmatrix}
	v_1''(x_1)& v_2''(x_1)&\cdots\\
	v_1''(x_2)& v_2''(x_2)&\cdots\\
	\vdots& \vdots&\ddots
\end{bmatrix}\cdot
\begin{bmatrix}
a_1\\
a_2\\
\vdots
\end{bmatrix}
=\begin{bmatrix}
-f(x_1)\\
-f(x_2)\\
\vdots
\end{bmatrix}$\qquad Das Gleichungssystem nach a aufl�sen

\subsection{Das Verfahren von Gauss (MSE)}

\textbf{Gausscher Matrize: }
$Q^{(n)}=\begin{bmatrix}
	Q_{1,1}& Q_{1,2}&\cdots\\
	Q_{2,1}& Q_{2,2}&\cdots\\
	\vdots & \vdots &\ddots\\
\end{bmatrix}$ \qquad mit \qquad $Q_{j,k}^{(n)}=\int\limits_{0}^{1}{v_j''(x)\cdot v_k''(x) dx}$\\
\textbf{Gausscher Vektor: } 
$q^{(n)}=\begin{bmatrix}
	q_1\\
	q_2\\
	\vdots\\
\end{bmatrix}$ \qquad mit \qquad $q_{k}^{(n)}=\int\limits_{0}^{1}{f(x)\cdot v_k''(x) dx}$\\

\textbf{L�sung nach Gauss:} $Q^{(n)}\cdot a+q^{(n)}=0\qquad \Rightarrow \qquad a=\textcolor{red}{\mathbf{-}}\left\{Q^{(n)}\right\}^{-1}\cdot q^{(n)}$

\subsection{Finite Elemente}

Die besprochenen Verfahren setzen die Wahl eines Satzes $v_1(x),\ldots,v_n(x)$ von Grundfunktionen voraus. Bei FEM wird mit lokalen Tr�gern (Grundfunktionen) gearbeitet, diese sind nur auf einem kleinen intervall ungleich Null. Der Vorteil dieses Vorgehens liegt darin, dass in einem Bereich nur einen Tr�ger die Approximationsfunktion beeinflusst. Der Nachteil liegt in der Anzahl der so ben�tigten Tr�ger, welche sehr hoch ist.


\subsubsection{Knotenvariablen}
\subsubsection{Formfunktionen}
\subsubsection{Elementmatrizen}
\subsubsection{h-Strategie}
\subsubsection{p-Strategie}

\subsection{FEM f�r elliptische PDEs}
\subsection{FEM f�r parabolische PDEs}