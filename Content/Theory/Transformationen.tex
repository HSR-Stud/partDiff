\subsection{Transformationen}
\todo{Beschreibung}


\begin{itemize}
\item Der Übergang von Funktionen zu Fourierreihen verwandelt eine partielle
Differentialgleichung in eine Familie gewöhnlicher Differentialgleichungen für
die einzelnen Fourier-Koeffizienten.
\item Integraltransformationen können ein partielle Differentialgleichung in eine
Familie partieller Differentialgleichungen mit weniger Variablen oder sogar
gewöhnlicher Differentialgleichungen verwandeln.
\item Integraltransformationen und die Rücktransformationen können Formeln
für die Lösungen gewisser partieller Differentialgleichungen liefern, und
damit die Frage beantworten, für welche Randwertvorgaben die Gleichungen
gut gestellt sind.
\end{itemize}

\subsubsection{Fourierreihe}

$\boxed{u(t,x)=\frac{a_0(t)}{2}+\sum\limits_{k=1}^{\infty}{a_k(t)\cos(kx)+b_k(t)\sin(kx)}}$\\[0.4cm]

\textbf{Beispiel:} Schwingende Saite: $\boxed{\partial_t^2u=\partial_x^2u}$

\begin{enumerate}
\item Ansatz der Fourieranalyse in PDGL einsetzen:\\
$$\partial_t^2(t,x)=\frac{a_0''(t)}{2}+\sum\limits_{k=1}^{\infty}{a_k''(t)\cos(kx)+b_k''(t)\sin(kx)} 
\qquad \qquad
\partial_x^2(t,x)=-\sum\limits_{k=1}^{\infty}{a_k(t)k^2\cos(kx)+b_k(t)k^2\sin(kx)}$$
$$\partial_t^2(t,x)=\partial_x^2(t,x) 
\qquad \Longleftrightarrow \qquad \frac{a_0''(t)}{2}+\sum\limits_{k=1}^{\infty}{a_k''(t)\cos(kx)+b_k''(t)\sin(kx)}=-\sum\limits_{k=1}^{\infty}{a_k(t)k^2\cos(kx)+b_k(t)k^2\sin(kx)}$$
$\boxed{\Rightarrow\quad \frac{a_0''(t)}{2}+\sum\limits_{k=1}^{\infty}{\big(a_k''(t)+a_k(t)k^2\big)\cos(kx)+\big(b_k''(t)+b_k(t)k^2\big)\sin(kx)}=0}$
\item Diese Gleichung ist nur lösbar wenn alle Koeffizienten verschwinden (Fourier-Theorie):\\[0.2cm]
$a_0''(t)=0 \qquad a_k''(t)=-k^2a_k(t)\qquad b_k''(t)=-k^2b_k(t)$
\item Durch die Fouriertransformation wurde die PDGL in ein DGL-System überführt, die Lösungen sind wohlbekannt:\\[0.2cm]
$a_0(t)=m_0(t)+c_0\qquad a_k(t)=A_k^a\cos(kt)+B_k^a\sin(kt)\qquad b_k(t)=A_k^b\cos(kt)+B_k^b\sin(kt)$
\end{enumerate}


\subsubsection{Anfangsbedingungen}
Die Differentialgleichungen für die Koeffizienten $ak(t)$ und $bk(t)$ können erst dann
vollständig gelöst werden, wenn Anfangs oder Randbedingungen gegeben sind.\\
\begin{itemize}
\item Anfangsbedingungen für Wellengleichung:\\
\quad $u(0,x)=f(x)\qquad \partFrac{u}{t}=g(x)$
\item Die Funktionen f und g können auch als Fourrierreihe dargestellt werden:\\
$f(x)=\frac{a_0^f}{2}+\sum\limits_{k=1}^{\infty}{a^f_k\cos(kx)+b^f_k\sin(kx)}$\\[0.2cm]
$g(x)=\frac{a_0^g}{2}+\sum\limits_{k=1}^{\infty}{a^g_k\cos(kx)+b^g_k\sin(kx)}$
\item Zusammen mit dem Ansatz für $u(t,x)$ ergeben sich die Gleichungen (für $t=0$):\\
$\frac{a_0(0)}{2}+\sum\limits_{k=1}^{\infty}{a_k(0)\cos(kx)+b_k(0)\sin(kx)}=\frac{a_0^f}{2}+\sum\limits_{k=1}^{\infty}{a_k^f\cos(kx)+b^f_k\sin(kx)}$\\[0.2cm]
$\frac{a'_0(0)}{2}+\sum\limits_{k=1}^{\infty}{a_k'(0)\cos(kx)+b_k'(0)\sin(kx)}=\frac{a_0^g}{2}+\sum\limits_{k=1}^{\infty}{a_k^g\cos(kx)+b^g_k\sin(kx)}$
\item Koeffizientenvergleich ergibt:\\
$a_k(0)=a_k^f\qquad a_k'(0)=a_k^g\qquad b_k(0)=b_k^f\qquad b_k'(0)=b_k^g$
\item Die vollständige Lösung ist damit:\\
$u(t,x)=\frac{a_0^g(t)+a_0^f}2+\sum\limits_{k=1}^{\infty}{\left(a_k^f\cos(kt)+\frac 1k a_k^2\sin(kt)\right)\cos(kx)+\left(b_k^f\cos(kt)+\frac 1k b_k^2\sin(kt)\right)}\sin(kx)$
\end{itemize}

\subsubsection{Inhomogene Wellengleichung}

Das Verfahren lässt sich auch auf die inhomogene Wellengleichung verallgemeinern. Das Störglied wird dabei ebenfalls als Fourierreihe entwickelt.

$\partial_t^2u-\partial_x^2u=f \qquad \Rightarrow \qquad f(t,x)=\frac{a_0^f(t)}{2}+\sum\limits_{k=1}^{\infty}{a^f_k(t)\cos(kx)+b^f_k\sin(kx)}$



\subsubsection{Laplace-Transformation}

$\boxed{F(t)=\int\limits_{0}^{\infty}{f(t)\e^{-st}} dt}$ \qquad Siehe auch weiter hinten in der Zusammenfassung!\\

\textbf{Lösung einer ODGL:}\\

$\dot{x}(t)+p x(t)=f(t) \qquad f(t)=q$\\
$\dot{x}(t)+p x(t)=f(t)\FT s X(s)-x(0)+pX(s)=F(s) \qquad f(t)\FT F(s)=\frac{q}{s}$\\

$\Rightarrow X(s)=\frac{F(s)+x(0)}{s+p}=\frac{q+x(0)}{s(s+p)}\Big|_{x(0)=0}\IFT x(t)=\frac{q}{p}(1-\e^{-pt})$\\

\textbf{Lösung einer PDGL:}\\

$\partFrac{u}{t}+x\partFrac{u}{x}=x\qquad t\geq 0,\quad x\geq 0\qquad u(x,0)=0,\quad u(0,t)=0\qquad x,t>0$\\

Transformation: $\partFrac{u}{t}+x\partFrac{u}{x}=x\FT sU(s,x)-u(x,0)+x\partFrac{U(s,x)}{x}=\frac{x}{s}\qquad \Rightarrow \qquad U(s,x)=\frac{x}{s(s+1)}$\\
$U(s,x)\IFT x(1-\e^{-t})$











