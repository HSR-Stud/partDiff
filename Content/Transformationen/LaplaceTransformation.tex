\section{Laplace Transformation}
	$$\boxed{F(s)=\int\limits_0^\infty f(t)e^{-st}dt} \qquad s=\sigma +j\omega$$\\
	- Definitionsbereich nur für \textbf{kausale} Systeme $\boxed{t\geq 0}$\\
	- Integrierbar über das Intervall $(0,\infty)$\\
	- Wachstum kleiner als der von eienr Exponentialfunktion $\boxed{\sigma > 0}$\\
	- $\sigma$ ist der Dämpfungsfaktor: $e^{-s}=e^{-\sigma} \cdot e^{-j\omega}$ \\
	- Fourier-Transformierte $F(\omega)$ kann durch die
	Laplace-Transformation $F(s)$ ausgedrückt werden.  \\
	- Fourier $\longleftrightarrow$ Laplace Umwandlungen nur wenn Polstelle
	($\sigma > 0$) dh. links von $j\omega$ Achse und kausal!
  
 	\subsection{Eigenschaften}
 	    \label{sec:Laplace Umwandulungen}
  		\renewcommand{\arraystretch}{2}
		\begin{tabular}{|l|l c p{5.5cm}|}
        	\hline
        	Linearität & 
 			$\alpha\cdot f(t) + \beta\cdot g(t)$ & $\FT$ & $\alpha\cdot F(s) +
 			\beta\cdot G(s)$ \\
 			\hline
 			Verschiebung im Zeitbereich &
 			$f(t\pm t_0) $ & $\FT$ & $ F(s)e^{\pm t_0 s}$ \\
 			\hline
 			Dämpfung (Verschiebung im Frequenzbereich) &
 			$f(t)e^{\mp\alpha t}$ & $\FT$ & $F(s\pm\alpha)$ \\
 			\hline
 			"Ahnlichkeit&
 			$f(\alpha t)$ & $\FT$ & $\frac{1}{\alpha}F \left (\frac{s}{\alpha} \right )
 			\quad 0 <\alpha \in\mathbb{R}$ \\
 			\hline
 			Faltung im Zeitbereich &
 			$f(t) \ast g(t)$ & $\FT$ &
 			$F(s) \cdot G(s)$\\
 			\hline 			
 			Faltung im Frequenzbereich &
 			$f(t) \cdot g(t)$ & $\FT$ & $\frac{1}{2 \pi j} F(s) \ast G(s)$ \\
 			\hline
 			Differentiation im Zeitbereich &
 			$f'(t)$ & $\FT$ & $sF(s) - f(0+)$ \\ 
 			&
 			$f''(t)$ & $\FT$ & $s^2 F(s) - sf(0+) - f'(0+)$\\ 
 			&
 			$f^{(n)}(t)$ & $\FT$ & $s^nF(s) - s^{n-1}f(0+) - s^{n-2} f'(0+) - \ldots
 			-s f^{(n-2)}(0+) - f^{(n-1)}(0+)$ \\
 			\hline
 			Diffrentation im Frequenzbereich &
 			$(-t)^n f(t)$ & $\FT$ & $F^{(n)}(s)$ \\
 			\hline
 			Integration &
 			$\int\limits_0^t f(\tau)d\tau$ & $\FT$ & $\frac{F(s)}{s}$ \\
 			\hline
 			Anfangswert &
 			$\lim_{t\rightarrow 0} f(t)$ muss exist. & $=$ & $\lim_{s\rightarrow
 			\infty} sF(s)$ \\
 			\hline
 			Endwert &
 			$\lim_{t\rightarrow \infty} f(t)$ muss exist. & $=$ & 
 			$\lim_{s\rightarrow 0} sF(s)$ \\
 			\hline
       	\end{tabular}
		\renewcommand{\arraystretch}{1}
		
		\subsection{Von Laplace zu Fourier}
			$s \rightarrow j\omega$ \hspace{0.5cm}
			Dies kann nur gemacht werden wenn Polstelle ($\sigma > 0$) links von
			$j\omega$-Achse ist und das System kausal ist.
			
		\subsection{Rüktransformation (Komplexe Integration)}
			$$f(t)=\int\limits_{x-j\infty}^{x+j\infty}F(s) \cdot e^{st} \cdot ds$$
			
		\subsection{Vorgehen Rücktransformation}
		\begin{tabular}{ll}
  			1. Ansatz & Versuchen Zähler Gleichnamig mit Nenner machen un danach
  			kürzen (Korrekturen!) \\
  			2. Ansatz & Partitialbruchzerlegung
		\end{tabular}
			
		\newpage
		
		\subsection{Rücktransformation über Tabelle}
			\begin{center}
				\let\DS=\displaystyle

{ \[
\begin{array}{|@{\hspace{1cm}}c@{\hspace{2cm}}c@{\hspace{2cm}}c@{\hspace{1cm}}|}
\hline &&\\
\sigma(t) & \FT & \DS\frac{1}{s} \\
&&\\
\hline &&\\
\sigma(t)\cdot t & \FT & \DS\frac{1}{s^2} \\
&&\\
\hline &&\\
\sigma(t)\cdot t^2 & \FT & \DS\frac{2}{s^3} \\
&&\\
\hline &&\\
\sigma(t)\cdot t^n & \FT & \DS\frac{n!}{s^{n+1}} \\
&&\\
\hline &&\\
\sigma(t)\cdot e^{\,\alpha\,t} & \FT & \DS\frac{1}{s-\alpha} \\
&&\\
\hline &&\\
\sigma(t)\cdot t\cdot e^{\,\alpha\,t} & \FT & \DS\frac{1}{(s-\alpha)^2} \\
&&\\
\hline &&\\
\sigma(t)\cdot t^2\cdot e^{\,\alpha\,t} & \FT & \DS\frac{2}{(s-\alpha)^3} \\
&&\\
\hline &&\\
\sigma(t)\cdot t^n\cdot e^{\,\alpha\,t} & \FT & \DS\frac{n!}{(s-\alpha)^{n+1}} \\
&&\\
\hline &&\\
\sigma(t)\cdot\sin\,(\omega\,t) & \FT & \DS\frac{\omega}{s^2+\omega^2} \\
&&\\
\hline &&\\
\sigma(t)\cdot\cos\,(\omega\,t) & \FT & \DS\frac{s}{s^2+\omega^2} \\
&&\\
\hline &&\\
\delta(t) & \FT & 1(s) \\
&&\\
\hline &&\\
\delta(t-a) & \FT & e^{-a\,s} \\
&&\\
\hline \end{array} \] }


			\end{center}
			\vfill
		
				